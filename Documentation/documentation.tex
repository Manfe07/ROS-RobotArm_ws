\documentclass[12pt,a4paper]{article}
\usepackage[utf8]{inputenc}
\usepackage[T1]{fontenc}
\usepackage{amsmath}
\usepackage[left=3cm,right=3cm,top=3cm,bottom=3cm]{geometry}
\usepackage{amsfonts}
\usepackage{amssymb}
\author{Manuel Fehren}
\title{automatisierter Roboterarm}
\begin{document}
\maketitle
\begin{center}
Ein simpler Servobetriebener 6-DOF Roboterarm, welcher durch ROS (Robot-Operation-System) automatisiert wurde, um zum Beispiel Schokolinsen mit Hilfe einer Kamera zu Automatisieren
\end{center}
\newpage
\tableofcontents
\newpage
\section{Einleitung}
Die eigentliche Idee war es automatisch kleine Gegenstände, zum Beispiel Schrauben oder Schoko-Linsen, mit Hilfe einer Kamera und eines Manipulators zu Sortieren. Dies sollten dann
\section{Wahl des Grundsystems}
\end{document}